\chapter{Introduction} \label{introduction}
%==========================================

XSB is a research-oriented, commercial-grade Logic Programming system
for Unix and Windows-based platforms.  In addition to providing nearly
all functionality of ISO-Prolog, XSB includes the following features:
\begin{itemize}
%
\item Evaluation of queries according to the Well-Founded Semantics
  \cite{VGRS91} through full SLG resolution (tabling with negation).
  XSB's tabling implementation supports incremental tabling, as well
  as call and answer subsumption.
%
\item A fully multi-threaded engine with thread-shared static code,
  and that allows dynamic code and tables to be thread-shared or
  thread-private.  This engine fully supports the draft ISO standard
  for multi-threading~\cite{Prolog-MT-ISO}.
%
\item Constraint handling for tabled programs based on an engine-level
  implementation of annotated variables and various costraint
  packages, including {\tt clpqr} for handling real constraints, and
  {\tt bounds} a simple finite domain constraint library.
%
  \item A package for Constraint Handling Rules \cite{Fruh} which can be
    used to implement user-written constraint libraries.
% 
\item A variety of indexing techniques for asserted code 
  including variable-depth indexing on several alternate arguments,
  fixed-depth indexing on combined arguments, trie-indexing.
%
\item A set of mature packages, to extend XSB to evaluate
  F-logic~\cite{KLW95} through the {\em FLORA-2} package (distributed
  separately from XSB), to model check concurrent systems through the
  {\em XMC} system, to manage ontologies through the {\em Cold Dead
    Fish} package, to support literate programming through the {\tt
    xsbdoc} package, and to support answer set programming through the
  {\tt XASP} package among other features.
%
\item A number of interfaces to other software systems, such a C, Java,
  Perl, ODBC, SModels \cite{NiSi97}, and Oracle.
%
\item Fast loading of large files by the {\tt load\_dync}
  predicate, and by other means.
%
\item A compiled HiLog implementation;
%
\item Backtrackable updates through XSB's {\tt storage} module that
  support the semantics of transaction logic \cite{BoKi94}.
%
\item Extensive pattern matching packages, and interfaces to {\tt libwww}
  routines, all of which are especially useful for Web applications.
%
\item A novel transformation technique called {\em unification
factoring} that can improve program speed and indexing for compiled
code; 
%
\item Macro substitution for Prolog files via the {\tt xpp}
  preprocessor (included with the XSB distribution).
%
\item Preprocessors and Interpreters so that XSB can be used to evaluate
  programs that are based on advanced formalisms, such as extended logic
  programs (according to the Well-Founded Semantics \cite{ADP94});
  Generalized Annotated Programs \cite{KiSu92}.
%
\item Source code availability for portability and extensibility under
  the GNU General Public Library License.
\end{itemize}
 
Though XSB can be used as a Prolog system,
%\footnote{Many of the Prolog
%  components of XSB were originally based on PSB-Prolog~\cite{Xu90},
%  which itself is based on version 2.0 of SB-Prolog~\cite{Debr88}.},
we avoid referring to XSB as such, because of the availability of SLG
resolution and the handling of HiLog terms.  These facilities, while
seemingly simple, significantly extend its capabilities beyond those
of a typical Prolog system. We feel that these capabilities justify
viewing XSB as a new paradigm for Logic Programming.  We briefly
discuss some of these features; others are discussed in Volumes 1 and
2 of the XSB manual, as well as the manuals for various XSB packages
such as FLORA, XMC, Cold Dead Fish, xsbdoc, and XASP.

\paragraph{Well-Founded Semantics} To understand the implications of
SLG resolution \cite{ChWa96}, recall that Prolog is based on a
depth-first search through trees that are built using program clause
resolution (SLD).  As such, Prolog is susceptible to getting lost in
an infinite branch of a search tree, where it may loop infinitely.
SLG evaluation, available in XSB, can correctly evaluate many such
logic programs.  To take the simplest of examples, any query to the
program:
\begin{center}
\begin{minipage}{3.8in}
\begin{verbatim}
:- table ancestor/2.

ancestor(X,Y) :- ancestor(X,Z), parent(Z,Y).
ancestor(X,Y) :- parent(X,Y).
\end{verbatim}
\end{minipage}
\end{center}
will terminate in XSB, since {\tt ancestor/2} is compiled as a tabled
predicate; Prolog systems, however, would go into an infinite loop.
The user can declare that SLG resolution is to be used for a predicate
by using {\tt table} declarations, as here.  Alternately, an {\tt
auto\_table} compiler directive can be used to direct the system to
invoke a simple static analysis to decide what predicates to table
(see Section~\ref{tabling_directives}).  This power to solve recursive
queries has proven very useful in a number of areas, including
deductive databases, language processing \cite{syntactica, semantica},
program analysis \cite{DRW96, CoDS96, Boul97}, model checking
\cite{RRRSSW97} and diagnosis \cite{GSTPD00}.  For efficiency, we
have implemented SLG at the abstract machine level so that tabled
predicates will be executed with the speed of compiled Prolog.  We
finally note that for definite programs SLG resolution is similar to
other tabling methods such as OLDT resolution~\cite{TaSa86} (see
Chapter
\ref{chap:TablingOverview} for details).

\begin{example} \label{ex:Russell}
The use of tabling also makes possible the evaluation of programs with
non-stratified negation through its implementation of the {\em
well-founded semantics} \cite{VGRS91}.  When logic programming rules
have negation, paradoxes become possible.  As an example consider one
of Russell's paradoxes --- 
the barber in a town shaves every person who does not shave himself ---
written as a logic program.
\begin{center}
\begin{verbatim} 
:- table shaves/2.

shaves(barber,Person):- person(Person), tnot(shaves(Person,Person)).
person(barber).
person(mayor).
\end{verbatim} 
\end{center}
Logically speaking, the meaning of this program should be that the
barber shaves 
the mayor, but the case of the barber is trickier.  If we conclude
that the barber does not shave himself our meaning does not reflect the 
first rule in the program.  If we conclude that the barber does shave
himself, we have reached that conclusion using information beyond what 
is provided in the program.  The well-founded semantics, does not
treat {\tt shaves(barber,barber)} as either true or false, but as {\em
undefined}. 
Prolog, of course, would enter an infinite loop.  XSB's treatment of
negation is discussed further in Chapter \ref{chap:TablingOverview}.
\end{example}

\index{multi-threading}
\paragraph{Multi-threading} From Version 3.0 onward, XSB has been thoroughly
revised to support multi-threading using POSIX or Windows threads.
Detached XSB threads can be created to execute specific tasks, and
these threads will exit when the query succeeds (or fails, or throws
an exception) and all thread memory reclaimed.  While a thread's
execution state is, of course, private, it shares many resources with
other threads, such as static code and I/O streams.  Dynamic code and
tables can be either thread-shared or thread-private by default or by
explicit declaration.

\paragraph{Constraint Support}
\index{attributed variables}
\index{Constraint Handling Rules}
%
XSB supports logic-based constraint handling at a low level through
attributed variables and associated packages (e.g. {\tt setarg/3}).  In
addition, constraints may be handled through Constraint Handling
Rules.  Constraint logic programs that use attributed variables may be
tabled; those that use Constraint Handling Rules may be efficiently
tabled if the {\tt CHRd} package is used.  Constraint programming in
XSB is mainly covered in Volume 2.

\index{indexing!star}
\paragraph{Indexing Methods} Data oriented applications may
require indices other than Prolog's first argument indexing.  XSB
offers a variety of indexing techniques for asserted code.  Clauses
can be indexed on a group of arguments or on alternative arguments.
For instance, the executable directive {\tt index(p/4,[3,2+1])}
specifies indexes on the (outer functor symbol of) the third argument
{\em or} on a combination of (the outer function symbol of) the second
and first arguments.  If data is expected to be structured within
function symbols and is in unit clauses, the directive {\tt
  index(p/4,trie)} constructs an indexing trie of the {\tt p/4}
clauses using a depth-first, left-to-right traversal through each
clause.  Representing data in this way allows discrimination of
information nested arbitrarily deep within clauses.  Advantages of
both kinds of indexing can be combined via {\em star-indexing}.
Star-indexing indicates that up to the first 5 fields in an argument
will be used for indexing (the ordering of the fields is via a
depth-first traversal).  For instance, {\tt index(p/4,[*(4),3,2+1])}
acts as above, but looks within 4th argument of {\tt p/4} before
examining the outer functor of argument 3 (and finally examining the
outer functors of arguments 2 and 1 together.  Using such indexing, XSB
routinely performs efficiently intensive analyses of in-memory
knowledge bases with millions of highly structured facts.  Indexing
techniques for asserted code are covered in Section \ref{sec:assert}.

\index{InterProlog} \index{ODBC Interface}
\paragraph{Interfaces} A number of interfaces are available to link XSB
to other systems.  In UNIX systems XSB can be directly linked into C
programs; in Windows-based system XSB can be linked into C programs
through a DLL interface.  On either class of operating system, C
functions can be made callable from XSB either directly within a
process, or using a socket library.  XSB can also inter-communicate
with Java through the InterProlog interface \footnote{InterProlog is 
available at \url{www.declarativa.com/InterProlog/default.htm}.} or
using YJXSB.  Within InterProlog, XSB and Java can be linked either through
Java's JNI interface, or through sockets.  XSB can access external
data in a variety of ways: through an ODBC interface, through an
Oracle interface, or through a variety of mechanisms to read data from
flat files.  These interfaces are all described in Volume 2 of this
manual.

\paragraph{Fast Loading of Code}  A further goal of XSB is to provide
in implementation engine for both logic programming and for
data-oriented applications such as in-memory deductive database
queries and data mining \cite{SaSw94}.  One prerequisite for this
functionality is the ability to load a large amount of data very
quickly.  We have taken care to code in C a compiler for asserted
clauses.  The result is that the speed of asserting and retracting
code is faster in XSB than in any other Prolog system of which we are
aware, even when some of the sophisticated indexing mechanisms
described above are employed.  At the same time, because asserted code
is compiled into SLG-WAM code, the speed of executing asserted code in
XSB is faster than that of many other Prologs as well.  We note
however, that XSB does not follow the ISO-semantics of
assert~\cite{LiOk87}.

\paragraph{HiLog} XSB also supports HiLog programming
\cite{ChKW93,SaWa95}.  HiLog allows a form of higher-order
programming, in which predicate ``symbols'' can be variable or
structured.  For example, definition and execution of {\em generic
  predicates} like this generic transitive closure relation are
allowed:
\begin{center}
\begin{minipage}{3.7in}
\begin{verbatim}
closure(R)(X,Y) :- R(X,Y).
closure(R)(X,Y) :- R(X,Z), closure(R)(Z,Y).
\end{verbatim}
\end{minipage}
\end{center}
where {\tt closure(R)/2} is (syntactically) a second-order predicate
which, given any relation {\tt R}, returns its transitive closure
relation {\tt closure(R)}.  XSB supports reading and writing of HiLog 
terms, converting them to or from internal format as
necessary (see Section~\ref{HiLog2Prolog}).  Special meta-logical
standard predicates (see Section~\ref{MetaLogical}) are also provided
for inspection and handling of HiLog terms.  Unlike earlier versions
of XSB (prior to version 1.3.1) the current version automatically
provides {\em full compilation of HiLog predicates}.  As a result,
most uses of HiLog execute at essentially the speed of compiled
Prolog.  For more information about the compilation scheme for HiLog
employed in XSB see~\cite{SaWa95}.

HiLog can also be used with tabling, so that the program above can also be
written as:
\begin{center}
\begin{minipage}{3.7in}
\begin{verbatim}
:- hilog closure.
:- table apply/3.

closure(R)(X,Y) :- R(X,Y).
closure(R)(X,Y) :- closure(R)(X,Z), R(Z,Y).
\end{verbatim}
\end{minipage}
\end{center}
as long as the underlying relations (the predicate symbols to which
$R$ will be unified) are also declared as Hilog.  For example, if {\tt
a/2} were a binary relation to which the {\tt closure} predicate would
be applied, then the declaration {\tt :- hilog a.} would also need to
be included.

\paragraph{Unification Factoring} For compiled code, XSB offers {\em
  unification factoring}, which extends clause indexing methods found
in functional programming into the logic programming framework.
Briefly, unification factoring can offer not only complete indexing
through non-deterministic indexing automata, but can also $factor$
elementary unification operations.  The general technique is described
in \cite{DRSS96}, and the XSB directives needed to use it are covered
in Section \ref{the_compiler}.

\index{Flora-2} \index{xsbdoc} \index{XASP}
\paragraph{XSB Packages} Based on these features, a number of
sophisticated packages have been implemented using XSB.  For instance,
XSB supports a sophisticated object-oriented interface called {\em
  Flora}.  {\em Flora} (\url{http://flora.sourceforge.net}) is
available as an XSB package and is described in its own manual,
available from the same site from which XSB was downloaded.  Another
package, XMC \url{http://www.cs.sunnysb.edu/~lmc} depends on XSB to
perform sophisticated model-checking of concurrent systems.  Within
the XSB project, the Cold Dead Fish package supports maintenance of,
and reasoning over ontologies; xsbdoc supports literate programming in
XSB, and XASP provides an interface to Smodels to support Answer Set
programming.  XSB packages also support Perl-style pattern matching
and POSIX-style pattern matching.  In addition, experimental
preprocessing libraries currently supported are Extended logic
programs (under the well-founded semantics), and Annotated Logic
Programs.  These latter libraries are described in Volume 2 of this
manual.

\section{Using This Manual}
We adopt some standard notational conventions, such as the name/arity
convention for describing predicates and functors, {\tt +} to denote
input arguments, {\tt -} to denote output arguments, {\tt ?} for
arguments that may be either input or output and {\tt \#} for
arguments that are both input and output (can be changed by the
procedure).  See Section \ref{mode_declarations} for more details.
\index{notational conventions}.  Also, the manual uses UNIX syntax for
files and directories except when it specifically addresses other
operating systems such as Windows.

Finally, we note that XSB is under continuous development, and this
document ---intended to be the user manual--- reflects the current
status (\version) of our system.  While we have taken great effort to
create a robust and efficient system, we would like to emphasize that
XSB is also a research system and is to some degree experimental.
When the research features of XSB --- tabling, HiLog, and Indexing
Techniques --- are discussed in this manual, we also cite documents
where they are fully explained.  All of these documents can be found
without difficulty on the web.

While some of \version\ is subject to change in future releases, we
will try to be as upward-compatible as possible. We would also like to
hear from experienced users of our system about features they would
like us to include.  We do try to accommodate serious users of XSB
whenever we can.  Finally, we must mention that the use of
undocumented features is not supported, and at the user's own risk.



%%% Local Variables: 
%%% mode: latex
%%% TeX-master: "manual1"
%%% End: 
