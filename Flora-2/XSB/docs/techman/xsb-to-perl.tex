\documentclass{article}
\begin{document}


\title{Introduction to XSB-Perl connectivity}
\author{Jin Yu and Michael Kifer}

\maketitle

There are many advantages in enabling the invocation of Perl functions from
within XSB. Perl can't be beat when it comes to text manipulation and a
large number of libraries have been written to facilitate processing Web
documents in Perl. XSB already has a flexible interface to Perl's pattern
matching and string substitution facilities and it is desirable to build
a more general interface. This section provides an overview of how this can
be done. For more details, see the manual pages that describe the subject:
%%
\begin{quote}
 man perlguts\\
 man perlembed\\
 man perlcall
\end{quote}
%%


\paragraph{Perl interface to C.}
To communicate with Perl, one would use a C program as a bridge.
Here we describe the Perl side of the story.

The Perl library is the collection of compiled C programs that are used to
start and communicate with create the interpreter. The C program that talks
to Perl must link to the Perl library. The Perl built-in library functions
perl{\_}alloc(), perl{\_}construct(), perl{\_}parse() and perl{\_}run() are
used to create and execute the Perl Interpreter Object.

After loading the Perl interpreter, the C program with
embedded Perl can be compiled into an executable program.
We illustrate the ideas using the example of Perl pattern matching
functions.

In the Perl language, the pattern matching syntax is as follows:

%%
\begin{quote}
 {\tt  m/<regular expression>/g(o)(i) }
\end{quote}
%%



The default input for pattern matching is the variable {\$}{\_}, and the
result of the match assigned to the match Perl variables {\$}1, {\$}2,
etc., the prematch substring {\$}`, the postmatch substring {\$}', the
exact match {\$}{\&}, and the match corresponding to the last parenthesized
regular subexpression, {\$}+. (Please see the Perl manual for the
explanation of these features.)

The first step is to learn some basics of the Perl internal data
structures. This discussion assumes some basic familiarity with Perl.

\begin{quote}
{\bf SV} - {\em Scalar Value\/}; stores int, float, and string data types.\\
{\bf AV} - {\em Array Value\/}; stores array data types.\\
{\bf HV} - {\em Hash Value\/}; stores the hash value types (also known as
associative arrays).
\end{quote}

As an example, consider a C function that implements pattern matching:

%%
\begin{quote}
    {\tt  int match(SV *string, char *pattern) }
\end{quote}
%%

This function searches the input {\em string\/} for a given Perl regular
expression, {\em pattern\/}, and returns the number of matches found.  The
code looks like:



\begin{verbatim}
int match(SV *string, char *pattern)
{
    ......
    SV *command = newSV(0);
    sv_setpvf(command, "$__text__='%s'; $__text__=~%s",
                       SvPV(string,na), pattern);
    perl_eval_sv(command, G_SCALAR);
    ......
}
\end{verbatim}


The library function {\em sv\_setpvf\/} takes a string and constructs a
Perl sentence where the first \%s is substituted with the value of {\em
  string\/} Here, {\em string\/} is of the Perl scalar type \verb|SV *|.
The function {\tt SvPV} converts this value back to the regular C pointer
to {\tt char (the flag {\tt na} tells SvPV tt to put a string terminating
  character at the end. The second \%s is replaced by the value of {\em
    pattern}.  As a result, the Perl variable \$\_\_text\_\_ is initialized
  with the value of {\em string\/}, and then it is matched against {\em
    pattern}. (The operator \verb|=~| performs pattern matching in Perl.)
  The actual evaluation of the command string is done using the library
  function {\tt perl\_eval\_sv}.  The flag {G\_SCALAR} means that we call
  the Perl subroutine in a scalar context ({\em context\/} is a special
  attribute of Perl environments; it is a source of much controversy and
  cannot be explained here).


  
\noindent
The following functions in C are used to allocate space for Perl scalar
values SV:

\begin{verbatim}
    SV* newSViv(IV); /* allocates integer scalar value */
    SV* newSVnv(double); /* allocates double-precision scalar value */
    SV* newSVpv(char*, int); /* allocates fixed-size string scalar value */
    SV* newSVpvf(const char*, ...); /*  allocates a list of strings */
    SV* newSVsv(SV*); /* allocates arbitrary scalar value */
\end{verbatim}


\noindent
The following functions assign C data to scalar values:

\begin{verbatim}
    void sv_setiv(SV*, IV); /* put integer into a scalar variable */
    void sv_setnv(SV*, double); /* put double value into a scalar variable */
    void sv_setpv(SV*, char*); /* put a string into a scalar variable */
    void sv_setpvn(SV*, char*, int); /* put a fixed size string */
    void sv_setpvf(SV*, const char*, ...); /* put a list of strings */
    void sv_setsv(SV*, SV*); /* put a scalar value into a scalar variable */
\end{verbatim}


The function
perl{\_}get{\_}sv() is used to get a Perl variable value and put it in the
SV-type structure. It returns a pointer to that SV data type. For
example:

\begin{verbatim}
    SV *returnValue = perl_get_sv( data, false);
\end{verbatim}

This takes Perl data and copies it into a newly allocated SV-type
structure; false is a flag.

\paragraph{The XSB side of the deal.}
The aforesaid function {\tt match()}  takes C strings, converts them into
Perl format, executes the match operation, and returns the result back to C.
To call this function from XSB, we need to take XSB strings, convert them
to C, call {\tt match()}, then get the result of {\tt match()} back to XSB.  

Here we briefly review the interface between C and XSB. First, C programs
called from XSB cannot contain {\tt main()}. These programs are compiled
into shared, dynamically loaded libraries, which are linked to XSB
executable when needed.

Just like with Perl, XSB data structures cannot be directly used in C, and
vice versa. Therefore, the following converter functions are provided:

\noindent
{\tt int ptoc\_int(int N)}: Get argument number N from an XSB predicate,
which is assumed to hold a Prolog integer, and return its integer value in
the valid C format.

\noindent
{\tt float ptoc\_float(int N)}: Get argument number N from an XSB predicate,
which is assumed to hold a Prolog floating point number, and return its
value in the valid C floating point format.

\noindent
{\tt char *ptoc\_string(int N)}: Get argument number N, which is assumed to
hold a Prolog atom, and return a pointer to the C string that corresponds
to this Prolog atom.

\noindent
{\tt void ctop\_int(int N, int V)}: Argument number N is assumed to hold a
free Prolog variable; this variable is then bound to the value V which must
be an integer.

\noindent
{\tt void ctop\_float(int N, float V)}: Argument number N is assumed to hold a
free Prolog variable, which is then is bound to a floating point value V.

\noindent
{\tt void c2p\_string(int N, char *V)}: Argument number N is assumed to hold a
free Prolog variable; this function binds that variable to the string
pointed to by V.


To illustrate how arguments are marshaled back and forth, consider
the following query:

\begin{verbatim}
    :- try_match( 'hello world', 'm/\wl/g' ).
\end{verbatim}


The C function \verb|try_match()| looks like this:

\begin{verbatim}
void try_match( void ) {
    char* string = ptoc_string(1);     /*get arguments*/
    char* pattern = ptoc_string(2);
    int result;
    ......
    result = match(string, pattern);
    ......
    return(result);                    /*output result*/
\end{verbatim}



Here, functions {\tt ptoc\_string(1)} and {\tt ptoc\_string(2)} pass the
first and the second arguments from XSB to the C function {\tt
  try\_match()}. These values are them passed to the function {\tt
  match()}.  As discussed earlier, the latter function passes these values
further to the Perl interpreter for evaluation. The return value then
eventually passed back to XSB. If the return value is non-0, then XSB
considers the built-in predicate {\tt try\_match()} as having executed
successfully. If it is 0, then the built-in is considered to have failed.


Sometimes it is necessary to pass values that are more complex than just
numbers and strings. XSB does not have an off-the-shelf function to marshal
such values to and from C. However, this can be accomplished by writing a
simple Prolog program that would pass around one scalar value at a time.
See the {\tt perlmatch} package for an example of how this can be done for
lists and arrays. 

\paragraph{Configuration issues.}
When compiling the buffer module to interface XSB to Perl, it is important
to compile it with the right Perl libraries. The difficulty here is that
Perl is compiled differently on different platforms, so there is no single
set of compile and load parameters that can be always used.

Fortunately, Perl can supply all the requisite information.
The following command provides the directory of the Perl libraries:

\begin{verbatim}
   perl -MConfig -e 'print $Config{archlib}'
\end{verbatim}
%% $

All that is needed is now needed is to add {\tt /CORE}, which is the actual
library. Let us denote this as {\tt Archlib}.  In addition to this, Perl
can tell the following about itself:

\begin{verbatim}
   perl -MConfig -e 'print $Config{cc}       (CC: the C-compiled)
   perl -MConfig -e 'print $Config{ccflags}' (CFlags: the C compiler flags)
   perl -MConfig -e 'print $Config{libs}     (CCLibs: the C libraries needed)
\end{verbatim}
%%$

\noindent
The XSB-Perl buffer module (which contains both {\tt match()} and {\tt
  try\_match()}) can then be compiled as follows:

\begin{verbatim}
   CC <CCFlags> -I<Archbib>  ...
\end{verbatim}

The loader will need the following flags: \verb|-L<Archbib> -lperl|.
See C interface section of the XSB manual for details on how to use these
flags.


\end{document}
